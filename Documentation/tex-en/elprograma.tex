\chapter{The LPMD code}
\label{chap:lpmd}

\section{Founds.}

The {\lpmd} code, start his developed during June in the 2007. This born like
an idea of implement molecular dynamics in a modular way, trying to be
user friendly  in the plugins and upgrade developing. the princpal developers
during the start of the has been: Sergio Davis, Claudia Loyola, Joaqu\'in
Peralta and Felipe Gonz\'alez, they are all from the \index{GNM}\textit{Grupo de
NanoMateriales} (\url{http://www.gnm.cl}) group. At this moment more
colaborators are join to us\footnote{Ver~cap.\ref{chap:auth}}.

The stable versions available of the software are:

\begin{itemize}
 \item Version 0.6.2 : Released November 1st 2011.
 \item Version 0.6.1 : Released January 3th 2010.
 \item Version 0.6.0 : Released June 12th 2009.
 \item Version 0.5.4 : Released January 1st 29th 2009.
 \item Version 0.5.3 : Released November 1st 2008.
 \item Version 0.5.2 : Released August 5th 2008.
 \item Version 0.5.1 : Released April 30th 2008.
 \item Version 0.5.0 : Released April 4th 2008.
 \item Version 0.4.0 : Released July 2007.
\end{itemize}

The upgrade system is controled by developers. From the 0.5 version of {\lpmd}
the software have three principal packages.

\begin{itemize}
\index{liblpmd} \item liblpmd \\
\index{API}Principal programming \textbf{API} developed by the team. This is
the core of all characteristics that have {\lpmd}. For a more detailed
description about this take a look in the apendix section ~\ref{ap:API}.
\index{lpmd-plugins} \item lpmd-plugins \\
Is a plugins set that has been developed, and incorporate characteristics from
the \textbf{API}, usually corresponend to Interatomic Potentials,
integrators, analyzers, or file managers.
\index{lpmd} \item lpmd \\
{\lpmd} Is the molecular dynamics (MD) software that use the plugins and the
API in order to realize the simulations. This include additional tools
like \textbf{lpmd-analyzer}, \textbf{lpmd-converter}, \textbf{lpmd-visualizer}
and in the last version \textbf{lpmd-plotter} that are used for the generation
of structures, analysis, visualization and image generation of atomic
configurations.
\end{itemize}

\section{Principal Goal}

The principal scheme of {\lpmd}, is the use of a control file
(\verb|file.control|), file that is load directly from the executable
\verb|lpmd| or other tools.

Consider by an example, a system of MD in that all options, like initial
temperature, simulation setp number, integrator, interatomic potential, atomic
positions, etc. are directly specified in a file ``\verb|simulation.control|'',
so {\lpmd} could be executed directly using, 

\begin{center}
 \texttt{lpmd simulation.control > simulation.output}
\end{center}
\noindent

In this way, {\lpmd} load all the neccesary information located in the file
\verb|simulation.control| and all output information is saved in the file
\verb|simulation.output|. Is important note that {\lpmd} even can generate
additional output file as modules or plugins had been loaded. From the version
0.6.1 during the {\lpmd} execution, you can see what plugins has been loaded.

\section{Characteristics}

Other characterisitc of {\lpmd} is that this have additional flags for the
control of the internal variables of a control file. By example, you can run
many different systems with different temperatures using a single control file.
Take the next line inside of a control file like as example

\begin{verbatim}
 ...
 prepare temperature t=$(TEMP)
 ...
\end{verbatim}
\noindent
with this, we can run {\lpmd} in order to assign to \verb|TEMP| variable a real
value, given directly by the command line, for example :

\begin{center}
 \texttt{lpmd -o TEMP=500 simulation.control > simulation.output}
\end{center}
\noindent
In this way the program start the simulation with a real temperature sistem of
500K.

The advantage of this characteristic is that we can reduce the difficult of
generate scripts for different simulations or even consecutive simulations of
molecular dynamics. By example with only one command we can run a set of
simulations with different values of a specific variable in a control file. For
a set of different temperatures, for example:

\begin{center}
 \texttt{for i in 300 400 500 ; \\do lpmd -o TEMP=\$i simulacion.control > simulacion-\$i.output ; done}
\end{center}
\noindent
This realize three simulations with different initial temperatures with just a
single control file.

Other interesting flag in {\lpmd} and their utilities is \verb|-p| that give
us information about the plugins that you have instaled in your system and that
are identified directly by {\lpmd}. You can check this using, for example:

\begin{center}
 \texttt{lpmd -p angdist}
\end{center}
\noindent
in this case \verb|angdist| is a plugin that calculate the angular distribution
function of a simulation cell. For more option of the executable try the
\verb|-h| option or \verb|man lpmd|. The next correspond to the standard help
information of lpmd using the \verb|-h| flag.

\begin{verbatim}
username@machine:~$lpmd -h
.....
================================================================================
===================== LPMD VERSION 0.X.Y =======================================
================================================================================
LPMD version 0.X.Y
Using liblpmd version M.N

Usage: lpmd [--verbose | -v ] [--lengths | -L <a,b,c>] [--angles | -A <alpha,beta,gamma>]
      [--vector | -V <ax,ay,az,bx,by,bz,cx,cy,cz>] [--scale | -S <value>] [--option | -O
      <option=value,option=value,...>] [--input | -i plugin:opt1,opt2,...]  
      [--output | -o plugin:opt1,opt2,...] [--use | -u plugin:opt1,opt2,...] 
      [--cellmanager | -c plugin:opt1,opt2,...] [--replace-cell | -r] [file.control]
       lpmd [--pluginhelp | -p <pluginname>]
       lpmd [--help | -h]
username@machine:~$ 
\end{verbatim}

In the next chapters of this manual we will give you a more detailed description
of other characteristics of {\lpmd} and their utilities.

%%%%%%%%%%%%%%%%%%%%%%%%%%%%%%%
